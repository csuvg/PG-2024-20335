El proyecto \textit{Señas Chapinas: Traductor de Lensegua} nació de la necesidad de reducir las barreras de comunicación que enfrenta la comunidad sorda en Guatemala, quienes utilizan la lengua de señas guatemalteca, Lensegua, como principal medio de expresión. Esta iniciativa tuvo el propósito de ofrecer una solución tecnológica que permita traducir Lensegua a texto y voz en tiempo real, ayudando a la comunidad sorda a participar en la sociedad de una manera más equitativa y autónoma.

Mi contribución al proyecto se centra en el desarrollo de la infraestructura de red y la arquitectura de backend, componentes esenciales para asegurar la estabilidad, seguridad, y eficiencia de la plataforma. Esta arquitectura se concibe como el núcleo técnico del sistema, integrando y facilitando la comunicación entre todos los módulos de la aplicación, desde el procesamiento de lenguaje natural hasta la visión por computadora. El backend no solo habilita el funcionamiento del traductor de señas, sino que también provee la base para escalar el proyecto en el futuro, adaptándose a las necesidades crecientes de la comunidad y del equipo de desarrollo.

Quiero expresar mi sincero agradecimiento al Ing. Miguel Novella, quien me brindó apoyo y orientación en cada etapa de esta compleja infraestructura. También agradezco a la Universidad del Valle de Guatemala, así como a En-Señas Guatemala y ASEDES, por los recursos y respaldo necesario para alcanzar los objetivos planteados. Su apoyo hizo posible que los sistemas de comunicación y procesamiento de datos del proyecto tomaran forma en condiciones óptimas y en un entorno seguro.

A mi familia y amigos, gracias por su inagotable apoyo y por ser mi motivación constante a lo largo de este camino. Este proyecto no habría sido posible sin su confianza y aliento inquebrantable.