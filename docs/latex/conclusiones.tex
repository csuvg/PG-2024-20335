En el desarrollo de este proyecto, se logró implementar un servidor robusto y eficiente que administra datos de manera segura. El uso de herramientas como Lynis permitió evaluar la seguridad del sistema, obteniendo un índice de robustez que, aunque podría mejorarse, ya coloca al servidor en un nivel seguro y confiable. Este servidor ha demostrado ser eficiente en pruebas de carga y extremo a extremo (E2E), permitiendo la ejecución de modelos de inteligencia artificial en tres máquinas virtuales distintas, lo que confirma su capacidad para soportar aplicaciones avanzadas de IA y visión por computadora.

Se implementaron configuraciones para abrir puertos y facilitar el acceso directo a las máquinas virtuales, permitiendo un acceso externo controlado. Además, se crearon scripts de despliegue para los modelos de inteligencia artificial y para el módulo central de APIs, optimizando el proceso de instalación y actualización de estos componentes. Este sistema de despliegue automatizado facilita las pruebas y la ejecución de modelos de IA, alineándose con el objetivo de crear un entorno accesible y funcional para usuarios externos.

Se levantaron tres máquinas virtuales completamente funcionales y accesibles, logrando virtualizar el servidor de manera exitosa. Estas máquinas permitieron probar las APIs con los modelos de IA cargados, obteniendo respuestas satisfactorias en cada caso. Este entorno virtual facilita la experimentación y el desarrollo de diferentes modelos de inteligencia artificial y visión por computadora, proporcionando a los usuarios un ambiente controlado y eficiente para el despliegue de sus aplicaciones.

Las pruebas de carga realizadas demuestran que el servidor mantiene un rendimiento sólido incluso bajo demandas elevadas. La combinación de Gunicorn y Nginx, junto con una base de datos gestionada por SQLAlchemy, contribuye significativamente a esta eficiencia, permitiendo manejar múltiples usuarios concurrentes sin comprometer la estabilidad. Estas pruebas confirman que la infraestructura desarrollada es capaz de manejar un flujo constante de solicitudes, optimizando el uso de recursos y reduciendo los costos operativos.

Uno de los objetivos principales era alcanzar un valor mínimo de 4.0 en la escala de CVE para seguridad, y los resultados obtenidos con Lynis superan esta meta, alcanzando un puntaje de 6.2. Este nivel de seguridad refuerza la protección contra vulnerabilidades y ataques, cumpliendo y excediendo las expectativas del proyecto. Con este logro, se garantiza que el servidor es capaz de operar en un entorno seguro y confiable, minimizando el riesgo de robos de datos y otros problemas de seguridad.
