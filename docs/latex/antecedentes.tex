No existen antecedentes específicos en la Universidad sobre proyectos previos que integren de forma directa la arquitectura universitaria con el fin de acceder a un servidor dedicado para pruebas y desarrollo de modelos de inteligencia artificial y visión por computadora. Este proyecto representa el primer esfuerzo en crear un ambiente accesible y seguro para el despliegue de modelos avanzados, permitiendo el acceso externo mediante una infraestructura de servidores segura y controlada.

Sin embargo, se puede mencionar como antecedente cercano el proyecto titulado \textit{Plataforma para el registro, visualización y difusión de competencias educativas para la mejora en el proceso de reclutamiento laboral de los miembros del departamento de computación de la UVG}, desarrollado por Paul De Jesús Belches Flores-Gómez y David Uriel Soto Alvarez. Este proyecto tiene una orientación hacia el uso de APIs para la centralización y gestión de competencias educativas en la Universidad, mejorando el acceso a la información y facilitando su uso en contextos administrativos y de reclutamiento laboral.

A nivel administrativo, el proyecto de Belches y Soto estableció un precedente en el uso de tecnologías API para centralizar y organizar datos en la Universidad, sentando una base técnica que influyó en el planteamiento de este proyecto al buscar una integración efectiva de APIs en el entorno universitario.

Por otro lado, el profesor Miguel Novella jugó un rol importante en apoyar los esfuerzos para establecer un acceso seguro a la arquitectura universitaria, un proceso que involucró la colaboración con el equipo de IT de la Universidad. Su experiencia y respaldo fueron esenciales en la gestión de permisos y en el acceso a la infraestructura universitaria, sentando las bases necesarias para la implementación exitosa del presente proyecto.
