Durante el desarrollo del proyecto, se buscó crear una infraestructura robusta y segura para la gestión de datos y la implementación de modelos de inteligencia artificial. En esta sección, se presentan los resultados clave y los aprendizajes derivados de las distintas fases del proyecto, enfocándonos en la eficiencia, seguridad y rendimiento del servidor, así como en la accesibilidad y la virtualización.

Durante la fase de implementación inicial, se configuraron tres máquinas virtuales, cada una diseñada para ejecutar modelos de inteligencia artificial y pruebas de carga. Este enfoque de virtualización demostró ser una solución efectiva para segmentar los entornos de desarrollo y producción, permitiendo un manejo más eficiente de los recursos computacionales. La utilización de herramientas como NGINX y Gunicorn en combinación con Flask para el backend resultó en una arquitectura capaz de soportar un alto volumen de solicitudes concurrentes, lo cual fue validado mediante pruebas de carga exhaustivas.

Los resultados obtenidos en las pruebas de carga reflejaron un manejo eficiente de las solicitudes hasta un nivel de 400-500 usuarios concurrentes sin mayores problemas de rendimiento. Estas pruebas, ejecutadas a través de un entorno controlado, permitieron observar métricas clave como el tiempo de respuesta, la utilización de CPU y la carga del sistema. El uso de SQLAlchemy para la gestión de bases de datos optimizó el acceso y manejo de datos, asegurando una interacción fluida entre las APIs y el servidor. Las métricas de rendimiento mostraron un incremento progresivo en el uso de recursos bajo cargas más altas, lo que resalta la necesidad de futuras optimizaciones para escenarios de carga extrema.

La prueba end-to-end (E2E) fue otra parte integral del proyecto, garantizando que el flujo completo de la aplicación funcionara como se esperaba. La prueba abarcó desde el registro de usuarios y la autenticación, hasta la interacción con funcionalidades avanzadas como el envío de videos y la gestión de traducciones. Los resultados indicaron un éxito del 100\%, demostrando que todos los módulos interactuaban correctamente y que la integridad de la aplicación estaba asegurada. Este éxito se tradujo en una validación de la funcionalidad y operatividad del sistema en condiciones reales de uso.

En cuanto a la seguridad, se realizaron múltiples auditorías usando Lynis, una herramienta reconocida en el análisis de vulnerabilidades y configuración de sistemas Linux. Los resultados iniciales arrojaron un ``Hardening Index'' de 58, que fue mejorado a 62 tras realizar ajustes y optimizaciones en la configuración del servidor. Este índice demuestra que el servidor es seguro, aunque siempre hay margen para futuras mejoras. Es importante destacar que los resultados obtenidos fueron superiores a la meta establecida de un índice de 4.0 en la escala de CVE, lo que subraya el éxito en la implementación de medidas de seguridad sólidas.

En cuanto a la seguridad, se realizaron múltiples auditorías utilizando Lynis, los resultados finales arrojaron un Hardening Index de 62, superando significativamente la meta inicial de 40, definida como un nivel adecuado para garantizar medidas de seguridad básicas. Es importante mencionar que este índice se obtuvo utilizando la configuración básica de Lynis, sin personalizar reglas ni omitir áreas de evaluación. Esto asegura que los resultados reflejen una auditoría rigurosa e integral del sistema, incluyendo configuraciones críticas que necesitaron atención. Si bien siempre hay margen para futuras mejoras, el puntaje alcanzado demuestra que el sistema implementó medidas de seguridad robustas, alineadas con los objetivos del proyecto y los estándares recomendados por Linux Audit.

El proyecto enfrentó desafíos notables, especialmente en la obtención de accesos y permisos necesarios para la integración de la infraestructura con la red universitaria. La colaboración con el equipo de TI y la gestión de permisos resultaron ser procesos complejos y burocráticos, que demandaron una planificación y coordinación significativa. Estos obstáculos, sin embargo, resaltaron la importancia de prever estas gestiones con anticipación para futuros proyectos.

En resumen, el desarrollo del servidor y su infraestructura demostró ser un éxito al cumplir con los objetivos de eficiencia, seguridad y accesibilidad. Las pruebas realizadas evidenciaron que el sistema puede soportar un alto volumen de carga y que sus componentes interactúan de manera efectiva. Las recomendaciones futuras incluyen continuar optimizando la seguridad y evaluar herramientas adicionales para el monitoreo y la mejora continua del rendimiento, asegurando que el sistema siga siendo confiable y eficiente a largo plazo.
