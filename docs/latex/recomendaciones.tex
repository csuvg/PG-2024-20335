En base a los resultados obtenidos y a las experiencias durante la implementación del proyecto, se proponen las siguientes recomendaciones para futuros trabajos y proyectos similares:

\begin{itemize}[leftmargin=*,labelsep=5mm]
    \item \textbf{Optimización de la virtualización del ambiente de desarrollo}: Actualmente, el proyecto utiliza un entorno virtualizado mediante contenedores para aislar y gestionar cada módulo o modelo de inteligencia artificial. Sin embargo, se pueden realizar mejoras adicionales, como la configuración avanzada de redes virtuales para facilitar la comunicación segura entre contenedores y el uso de herramientas como Terraform para la gestión de infraestructura como código, permitiendo una replicación más eficiente del entorno.

    \item \textbf{Refinamiento del uso de entidades ORM para bases de datos}: El proyecto integra un sistema de mapeo objeto-relacional (ORM) con SQLAlchemy, permitiendo una interacción estructurada y eficiente con la base de datos. No obstante, sería valioso explorar configuraciones avanzadas, como la implementación de estrategias de particionamiento horizontal en bases de datos para manejar grandes volúmenes de datos, y el uso de migraciones automatizadas con Alembic para garantizar la coherencia de esquemas durante actualizaciones.

    \item \textbf{Fortalecimiento del sistema de monitoreo y notificaciones}: Actualmente, el sistema cuenta con un monitoreo continuo utilizando herramientas como NGINX Amplify. A pesar de ello, podrían implementarse mejoras como la integración con Prometheus y Grafana para un análisis más profundo de métricas personalizadas y alertas basadas en patrones de uso. También es recomendable configurar un sistema de balanceo de carga dinámico que permita escalar en función de las métricas recolectadas.

    \item \textbf{Avance en la seguridad y el rendimiento del sistema}: Se han tomado medidas significativas para asegurar y optimizar el sistema mediante configuraciones avanzadas de NGINX y Gunicorn. No obstante, la habilitación de HTTP/2 para mejorar el rendimiento de la transferencia de datos y la integración de autenticación mTLS para comunicaciones internas son pasos adicionales que pueden aumentar el nivel de seguridad y rendimiento. Además, la implementación de pruebas de carga automatizadas con herramientas como Locust puede ayudar a evaluar la robustez del sistema frente a escenarios extremos.

    \item \textbf{Optimización en la implementación y configuración de Lynis}: Aunque Lynis ya ha demostrado ser una herramienta confiable para la auditoría de seguridad del sistema, se recomienda explorar configuraciones avanzadas que permitan adaptar su funcionamiento a las necesidades específicas del proyecto. Esto incluye la posibilidad de personalizar reglas para ignorar aspectos no relevantes en el entorno evaluado y priorizar áreas críticas mediante configuraciones específicas. Además, se sugiere establecer un ciclo de auditorías periódicas para identificar y mitigar nuevas vulnerabilidades, acompañado de un proceso de actualización continua de las configuraciones de seguridad. La flexibilidad de Lynis también permite ajustar el índice de fortalecimiento para optimizar su utilidad, alineándolo con los objetivos y prioridades del sistema. Según CIS Benchmarks, personalizar herramientas de auditoría es una práctica estándar que asegura la relevancia y precisión de las evaluaciones, permitiendo consolidar medidas robustas y prácticas de seguridad eficientes para entornos en evolución.
\end{itemize}

Estas recomendaciones están orientadas a consolidar y mejorar las capacidades existentes del proyecto, incluyendo seguridad, eficiencia y escalabilidad, al mismo tiempo que se asegura la portabilidad y facilidad de implementación en futuros entornos.
